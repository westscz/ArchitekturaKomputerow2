\chapter{Laboratorium 4}

\section{Tematyka}
Tematem laboratorium było zapoznanie się z działaniem FPU.
\section{Zakres prac}
Zadaniem było stworzenie trzech aplikacji wykorzystujących łączenie ASM i C:
\begin{enumerate}
	\item Program który odczyta i zinterpretuje dane z rejestru stanu (FPSR) oraz ustawi odpowiednie bity rejestru sterującego (FPCR)
	\item Program który wytwarza poszczególne wyjątki zmiennoprzecinkowe i wyświetla odpowiednie komentarze
	\item Program który pozwoli na przeprowadzenie operacji zmiennoprzecinkowych w dwóch wersjach. Pierwsza jako proste liczenie pola trójkąta, druga wykorzystująca iteracje (np. metoda Newtona-Raphsona)
\end{enumerate}
Przy każdym z zadań działania powinny być przeprowadzone w kodzie ASM. Cała reszta jak pobranie argumentów czy wyświetlenie wyników powinno być wykonane w C.

\section{Rozwiązanie}
\subsection{Odczytanie rejestru stanu}
Do odczytania rejestru stanu wykorzystano specjalne instrukcje. Następnie po uzyskaniu wartości statusu, należało zweryfikować jego poszczególne bity. Nas interesowało pierwsze sześć bitów. Aby tego dokonać w jak najprostszy sposób skorzystano z intrukcji "`shr"' która przesuwa bity o zadaną ilość bitów w prawo, przy przesuwaniu ich o jeden bit, ten trafiał do flagi TODO (flagi przeniesienia), a tą można zweryfikować instrukcją "`lea"'. W ten sposób sprawdzono każdy kolejny bit, i wyświetlono odpowiedni komunikat gdy natrafiono na przeniesienie.
\begin{lstlisting}[frame=single, basicstyle=\small, caption=Odczytanie i weryfikacja rejestru stanu]
  fstsw	statusword
  fwait
  mov	statusword, %bx

  mov $1, %cl
  and $0, %rax
check_operation:
  shr %rbx
  jnc check_denormal
  leaq err_operation, %rdi
  call printf
\end{lstlisting}
Ze względu na dopieszczenie kodu do perfekcji, zmieniając metody przetwarzania statusu, to zadanie zajęło zbyt dużą część laboratorium. Odczytanie rejestru sterującego byłoby rozwiązane w sposób podobny, jednakże tam uwagę należałoby zwrócić na dwie pary bitów: 11-10 oraz 9-8. Bity sprawdzamy tak jak powyżej, z pomocą instrukcji "`shr"'. 

\section{Wnioski}
Ze względu na problemy związane ze zrozumieniem działania jednostki zmiennoprzecinkowej oraz problemu związanego z pewnymi instrukcjami na zajęciach laboratoryjnych nie udało się ukończyć drugiej części zadania. Jest to na pewno jednostka bardzo przydatna do przeprowadzania dokładnych obliczeń, z wartościami po przecinku. Pisanie kodu w assemblerze pozwala nam także na zmianę precyzji tych obliczeń oraz sposobu zaokrąglania co także można uznać za plus.
