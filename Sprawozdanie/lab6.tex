\chapter{Laboratorium 6}
\section{Tematyka}
Tematem laboratorium było zapoznanie się z procesorem: wyłuskiwaniem informacji na jego temat, wykorzystywaniem cykli procesora oraz weryfikowania działania pamięci cache.
\section{Zakres prac}
Zadaniem było zapoznanie się z:
\begin{enumerate}
	\item CPUID - Wydobycie informacji na temat procesora i pamięci podręcznej.
	\item RDTSC - Odczyt aktualnego licznika cykli procesora.
	\item CACHE - Instrukcje z rozszerzeniem *"'FENCE"' a także wykonanie pętli z cyklicznym dostępem do kolejnych komórek tablicy.
\end{enumerate}
\section{Rozwiązanie}
\subsection{CPUID}
CPUID pozwala na odczytanie wielu informacji na temat procesora, są to przede wszystkim informacje na temat tego jakie instrukcje on obsługuje czy jakiego rozmiaru posiada pamięć cache i tym podobne. Dostać się do tych informacji można na dwa sposoby które w sumie sprowadzają się do jednego. Pierwszym sposobem jest wprowadzenie "`ID"' do rejestru "`EAX"' i wywołanie funkcji CPUID która zwróci odpowiednie dane do rejestrów EAX-EDX. 
\begin{lstlisting}[frame=single, basicstyle=\small, caption=Funkcja CPUID]
 mov $2, %eax
 cpuid
\end{lstlisting}

Drugim sposobem jest wykorzystanie w C funkcji "`\_\_get\_cpuid"' znajdującej się w bibliotece "`cpuid.h"'. Należy do niej wprowadzić "`ID"' oraz adresy do czterech zmiennych int które będą emulować nasze cztery rejestry.
\begin{lstlisting}[language=C, frame=single, caption=Funkcja odczytująca i weryfikująca cpuid, label=l:vertical, basicstyle=\small]  
#include <cpuid.h>
int main()
{
 int a,b,c,d,lvl;
 lvl = 1;
 __get_cpuid(lvl,&a,&b,&c,&d);
 printf("Floating-point unit on-Chip: %d\n", d & 0x1);
 printf("Intel Architecture MMX technology supported: %d\n", d>>23 & 0x1);
 printf("Streaming SIMD Extensions supported: %d\n", d>>25 & 0x1);
 printf("Streaming SIMD 2 Extensions supported: %d\n", d>>26 & 0x1);
 printf("CLFLUSH line size: %d\n", b>>8 & 0xF);
\end{lstlisting}
Tak jak wspomniano wcześniej, zasada działania jest tak naprawdę taka sama. W wyniku otrzymujemy ciągi 32 bitów które dla odpowiednich "`ID"' mają odpowiednie wartości. Dla przykładu w powyższym kodzie widać, iż dla jedynki możemy wyciągnąć informacje na temat wspieranych technologii, używanych na poprzednich laboratoriach.

\subsection{RDTSC}
Rdtsc jest funkcją która zwraca ilość cykli procesora. W ten sposób najefektywniej można porównywać czas pracy funkcji, algorytmów itp. rozwiązań. Wystarczy odczytać stan procesora przed funkcją, po funkcji i odjąć wartości aby otrzymać ilość cykli procesora potrzebną na wykonanie danego zadania. Na te potrzeby został stworzony poniższy kod asemblerowy aby otrzymywać informacje z RDTSC.
\begin{lstlisting}[frame=single, basicstyle=\small, caption=Funkcja RDTSC]
.global _rdtsc
_rdtsc:
 pushq %rbp
 movq %rsp, %rbp

 rdtsc

 movq %rbp, %rsp
 pop %rbp
ret
\end{lstlisting}

\subsection{Cache}
Pamięć cache pozwala na szybszy dostęp do danych. Gdy tylko procesor widzi, że pracujemy na tablicy, stara się przetrzymywać ją całą w pamięci aby umożliwić jak najszybsze przeprowadzanie operacji. Aby sprawdzić czy to prawda przeprowadzono eksperyment. Do testu stworzono kilka tablic o rozmiarach od dużo mniejszych do dużo większych od rozmiaru cache a także tablice która jest minimalnie od niego większa. Następnie puszczono operacje odczytu z tablicy taką samą ilość razy dla każdej z nich
\begin{lstlisting}[language=C, frame=single, caption=Przykładowa funkcja obliczająca ilość cykli dla odczytu danych z tablicy 4096 intów, label=l:vertical, basicstyle=\small]  
int tablica[4096];
unsigned long start, stop;
int loop=100000;
start = _rdtsc();
for(i=0;i<loop;i++){
	tablica[i%4096];}	
stop = _rdtsc();
printf("Time for 4096 bytes:	%lu\n",stop-start);
\end{lstlisting}
Z tego eksperymentu wynikło, że dla tablic które mieściły się w pamięci cache, czas odczytu wahał się w okolicach podobnej ilości cykli. Dla tablicy która była niewiele większa odnotowano najdłuższy czas dostępu, a dla kolejnych były to ilości pomiędzy poprzednimi wartościami.

\section{Wnioski}
Dzięki laboratorium zapoznano się z działaniem pamięci cache, sposobem testowania czasu działania algorytmów oraz odczytywania danych z procesora. Jest to przydatna wiedza, gdyż pozwoli usprawnić sposób pisania aplikacji, weryfikowania czasu ich działania. Najfajniejsza okazała się funkcja CPUID dzięki której możemy tworzyć aplikacje które same odczytają istotniejsze informacje o naszym procesorze i wykorzystają jego maksymalne możliwości.