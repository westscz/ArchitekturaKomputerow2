\chapter{Laboratorium 1}
\section{Tematyka}
Tematem laboratorium było zapoznanie się z podstawowymi działaniami z wykorzystaniem
instrukcji assemblerowych. Głównym celem zajęć było przetwarzanie ciągów znaków.
\section{Zakres prac}
Zadaniem było stworzenie trzech aplikacji.
\begin{enumerate}
	\item Program który zamieniałby, w ciągu znaków, znaki duże na małe oraz małe na duże
	\item Program który implementowałby prosty szyfr cezara
	\item Program który szyfrował i deszyfrowałby liczby całkowite (wykorzystując program z zadania drugiego)
Program szyfrujący miałby zostać stworzony w wariancie 32 oraz 64 bitowym.
\end{enumerate}
\section{Rozwiązanie}
\subsection{Zamienianie znaków}
Pierwsze zadanie nie było trudne. Dzięki obszernej instrukcji do zajęć laboratoryjnych bezproblemowo udało się wykonać ten prosty typ programu. Poniższy kod jest odpowiedzialny za pobranie, zmiane rozmiaru i ponowne zapisanie znaku.
\begin{lstlisting}[frame=single, basicstyle=\small, caption=Funkcja zmieniająca rozmiar znaków]
function:
 movb in(,%esi,1), %al
 cmp $END_OF_LINE, %al
 je end
 cmp $'Z', %al
 jle big
 sub $DISTANCE, %al
 movb %al, out(,%esi,1)
 inc %esi
 jmp function

big:
 add $DISTANCE, %al
 movb %al, out(,%esi,1)
 inc %esi
 jmp function
\end{lstlisting}
Nie jest to najbardziej optymalny typ rozwiązania. Z aktualną wiedzą stwierdzam, że wystarczyłoby na każdej z liter użyć funkcji XOR z odpowiednią maską która zamieniałaby jeden bit na odwrotny, który odpowiada za rozmiar litery. Takie rozwiązanie zostało wprowadzone podczas pisania funkcji XOR'ującej na kolejnych laboratoriach.
\subsection{Szyfr cezara}
Szyfr cezara to jeden z najbardziej znanych typów szyfrowania. Pierwszy znak wyznacza przesunięcie znaków w alfabecie. Przyjęto, iż znak "`A"' oznacza brak szyfrowania. Przyjęto także, że duży pierwszy znak ma powodować szyfrowanie, a mały znak deszyfrowanie. A co dalej następuje, pierwszy znak należy najpierw wyciągnąć i zweryfikować z czym mamy do czynienia.

\begin{lstlisting}[frame=single, basicstyle=\small, caption=Rozpoznanie znaku]
Sign:
 mov $0, %ecx
 movb in(,%esi,1), %cl

 cmp $'A', %cl
 jl end
 cmp $'z', %cl
 jg end
 cmp $'Z', %cl
 jle Decryption

Encryption:
 sub $'a', %ecx
eFunction:
 inc %esi
 movb in(,%esi,1), %al
 cmp $END_OF_LINE, %al
 je end
\end{lstlisting}

Po zapoznaniu się ze znakiem i typu działania, rozpoczynamy działanie na reszcie ciągu znaków. Każdy ze znaków należy przesunąć o odpowiednią wartość w lewo lub prawo, a następnie zweryfikować, czy nie wyszliśmy poza zakres liter. W przypadku przekroczenia zakresu liter, należy odjąć litery, aby doprowadzić do zapętlenia.

\subsection{Szyfrowanie liczb}
Z szyfrowaniem liczb nie było dużego problemu. Postawione założenie mówi, iż gdy w ciągu znaków zostaną odnalezione cyfry, należy je poprzedzić znakiem "`X"', a każdą z cyfr potraktować jak kolejne znaki alfabetu. Dla przykładu dla liczby "`1994"' należałoby przeprowadzić konwersje na "`xAIID"'. Budzi to pewne obawy, gdyż przy dekodowaniu po trafieniu na znak "`X"' powinniśmy znaki z zakresu od A do I traktować jako liczby, a co jeśli to już będzie zakodowana treść a nie liczba? Tak więc do kodu z zadania drugiego została dorobiona funkcja która sprawdza nasz ciąg znaków i w miarę potrzeby zamienia wykryte w nim liczby według powyższego schematy 

\begin{lstlisting}[frame=single, basicstyle=\small, caption=Szyfrowanie liczb całkowitych]
 movb text(,%esi,1), %cl
 cmp $END_OF_LINE, %cl
 je trEnd
 cmp $'0', %cl
 jl trSave
 cmp $'9', %cl
 jg trSave

trNumberStart:
 movb $'X', in(,%edi,1)
 inc %edi

trNumber:
 subb $NUMBER, %cl
 addb $'A', %cl
 movb %cl, in(,%edi,1)
 inc %esi
 inc %edi
\end{lstlisting}

\section{Wnioski}
Pisanie w assemblerze nie jest prostym tematem. Należy zwracać uwagę na wiele szczegółów na które przy pisaniu chociażby w C nikt nie zwraca uwagi. Najtrudniej było rozpocząć pracę z instrukcjami, pamiętać co dzieje się z rejestrami w przypadku niektórych instrukcji. W przypadku pisania aplikacji pod system x86, przy małej ilości rejestrów bardzo pomocny okazał się bufor i stos. Inny problem sprawiło późniejsze zrozumienie co dzieje się w kodzie, gdzie z pomocą przyszedł debugger GDB. Dzięki opanowaniu tego narzędzia z powodzeniem można było przejrzeć krokowo zmiany w pamięci i rejestrze.